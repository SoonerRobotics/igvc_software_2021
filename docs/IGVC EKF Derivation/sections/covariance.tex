\section{Covariance and Uncertainty}

Covariance is the relationship that every variable has with every other variable, and it is very important in multidimensional systems. We have our main covariance $\boldsymbol{P}$, also called the estimate uncertainty, which is modified both in the \textit{Predict} and in the \textit{Update} phases. 

In the \textit{Predict} step, we extrapolate the estimate uncertainty to create a prediction for $\boldsymbol{P}$ at the next timestep. This is done by using our state transition matrix previously obtained from the motion model. 
\begin{equation}
    \boldsymbol{P}_{k+1} = \boldsymbol{F} \cdot \boldsymbol{P} \cdot \boldsymbol{F}^T + \boldsymbol{Q}
\end{equation}
where $\boldsymbol{Q}$ is a matrix representing the process noise. This is hard to nail down and understand physically, but think of it as the uncertainty in the process of the EKF itself. There are formulas and ways to find more complicated versions of all entries, but it suffices in our case to simply use the 8x8 identity matrix multiplied by a constant that we can tune. This constant tends to remain in the range (1.1, 1.3) in our application.

In addition to these uncertainties in the estimations and in the process, we of course have uncertainties in our measurements. If sensors give direct values for their variance, those should be used. We define the measurement uncertainty as a 6x6 matrix, $\boldsymbol{R}$, with the variance of each measurement as the entry on the diagonal corresponding to that measurement's position in $\boldsymbol{z}$. We know that our approximations for the positions using GPS coordinates are fairly scuffed and have a standard deviation around 5 meters, whereas our IMU and encoders can easily be more accurate than a tenth of a unit. As such, we used the following matrix as a starting point. We frequently modify these values on the fly based on which sensors are performing well and which ones are having more interference than expected. Check the competition build of the code to see what we finalized at the venue.

\begin{equation}
    \boldsymbol{R} = 
    \begin{pmatrix}
    35 & 0 & 0 & 0 & 0 & 0 \\
    0 & 25 & 0 & 0 & 0 & 0 \\
    0 & 0 & 0.1 & 0 & 0 & 0 \\
    0 & 0 & 0 & 0.1 & 0 & 0 \\
    0 & 0 & 0 & 0 & 0.01 & 0 \\
    0 & 0 & 0 & 0 & 0 & 0.01
    \end{pmatrix}
\end{equation}

Note that if a sensor is being absolutely horrible or breaks it can be essentially disabled by setting its uncertainty value in this matrix to something huge like 100,000. Even if the filter never receives data from a sensor (i.e., it is unplugged or not functioning), it will use the starting value, 0, weighted with a massive uncertainty; this means the filter will still work if a sensor is disconnected, albeit not as well as with all the expected data.